%\pdfoutput=1
\documentclass[leqno,dvipsnames,11pt]{article}
\usepackage[minionmath]{mystyle}
\usepackage{standard_macros}
\usepackage[margin=1in]{geometry}
\DeclareMathOperator{\Nfib}{NFib}
\DeclareMathOperator{\Fib}{Fib}
\DeclareMathOperator{\DiscFib}{DiscFib}
\DeclareMathOperator{\splitfib}{split}
\DeclareMathOperator{\Arr}{Ar}
\DeclareMathOperator{\target}{cod}
\DeclareMathOperator{\targ}{target}
\DeclareMathOperator{\const}{const}

\newcommand{\Path}{\mathrm{Path}}
\newcommand{\Ecat}{\catname{E}}
\newcommand{\Ccat}{\catcal{C}}
\newcommand{\Pcat}{\catcal{P}}
\newcommand{\Qcat}{\catname{Q}}
\newcommand{\Dcat}{\catname{D}}
\newcommand{\Bcat}{\catname{B}}
%\newcommand{\cC}{$\Ccal$}
\newcommand{\cP}{\Pcal}
\newcommand{\isoname}[1]{\overset{#1}{\cong}}
\newcommand{\Core}[1]{\mathrm{Core}\paren{#1}}

\newcommand{\markedscat}[1]{\paren{{#1}, E^1_{#1},E^2_{#1}}}
\newcommand{\walkingadjoint}{\catcal{E}^{\textrm{adj}}}
\newcommand{\suspension}[1]{\Sigma\catcal{#1}}

\newcommand{\oversetlongarrow}[1]{\overset{#1}{\longrightarrow}}


%Priorities
%
\newcommand{\lowprio}{\textbf{(Priority: \textcolor{blue}{Low}})}
\newcommand{\medprio}{\textbf{(Priority: \textcolor{yellow}{Medium}})}
\newcommand{\hightprio}{\textbf{(Priority: \textcolor{red}{Hight}})}


%%%%%%% comment bubbles %%%%

%% Santiago's notes
\newcommand{\stnote}[1]{\todo[color=cyan!40,linecolor=cyan!40!black,size=\tiny]{#1}}
\newcommand{\stnoteil}[1]{\ \todo[color=cyan!40,linecolor=cyan!40!black,size=\normalsize]{#1}}




\renewcommand{\contentsname}{\large Contents}
\usepackage{citationscheme}
\usepackage{multirow}
\usepackage{algorithm}
\usepackage{algpseudocode}
\usepackage{listings}
\usepackage{xparse}
\usepackage{fontawesome5}
\usepackage{showlabels}
\usepackage{booktabs}


% Optional but recommended styling
\lstset{
	language=Python,
    breaklines=true,
    basicstyle=\ttfamily\small,
    keywordstyle=\color{blue},
    commentstyle=\color{green!60!black},
    stringstyle=\color{purple},
	otherkeywords={self,True,False, keras, torch},
    numbers=left,
    numberstyle=\tiny,
    numbersep=1pt,
	numberstyle=\tiny\color{gray},
    showstringspaces=false,
    keepspaces = true,
    tabsize=2,
    literate={\ \ }{{\ }}1,
    captionpos=b
}

% Python-specific settings
\lstdefinestyle{python}{
    language=Python,
    morekeywords={import,from,class,def,for,while,if,is,in,elif,else,try,except,finally,print,exec,with,as ,lambda},
    emph={self},
    emphstyle=\color{red}
}

%first parameter is the subfolder name, eg classification, GLM, etc. Second is the python file name, eg logistic_regression.py, and third is the function name, eg fit. The function will be extracted from the python file and displayed in the document.
\NewDocumentCommand{\script}{mmm o}{%
  \immediate\write18{python ../../extract_function.py ../../Implementations/#1/#2 #3 > ../../temp_function.py}%
  \lstinputlisting[language=Python, #4]{../../temp_function.py}%
}

\addbibresource{references.bib}

%-------------------------------------------------------------------%
%-------------------------------------------------------------------%
% Title data                                                        %
%-------------------------------------------------------------------%
%-------------------------------------------------------------------%


\title{\Large \textbf{ \textsc{Machine Learning from scratch}}}

\author{\textsc{Santiago Toro}}
%\author[2]{\textsc{Name 2}}
%\author[3]{\textsc{Name 3}}
%\author[4]{\textsc{Name 4}}
%\author[5]{\textsc{Name 5}}
%\author[6]{\textsc{Name 6}}
%
%\affil[1]{Université Bretagne Occidentale}
%\affil[2]{Univ 2}
%\affil[3]{Univ 3}
%\affil[4]{Univ 4}
%\affil[5]{Univ 5}
%\affil[6]{Univ 6}

\date{\normalsize \today}


\begin{document}
	
\maketitle

\begin{abstract}
	\lipsum[1]
\end{abstract}
	
\setcounter{tocdepth}{4}
\setcounter{secnumdepth}{4}
\tableofcontents
\cleardoublepage	

	%Introduction
%	\setcounter{section}{-1}

	%FIRST PART%%%%%%%%%%%%%%%%%%%%%%%%%%%%
	\part{Preliminaries}
	\section{Introduction}

What is "Learning"? Types of learnings, etc. 
	\section{Multivariate Calculus}
	\section{Probability Theory}

\subsection{Probabilities}

\subsection{Probability Distributions}
	%%%%%%%%%%%%%%%%%%%%%%%%%%%%%%%%%%%%%%%%
	%SECOND PART
	\cleardoublepage
	\part{Supervised Learning}
	\section{Regression}

\subsection{Linear Regression}

In statistical modeling, regression analysis consists of a set of statistical processes and techniques for estimating the relationships between a dependent variable (often called the outcome or response variable, or a label in machine learning) and one or more error-free independent variables (often called regressors, predictors, covariates, explanatory variables or features). 

The goal of regression is then to predict the value of one or more continuous response variables $y$ given the values of a $m$-dimensional vector $\vec{x}$ of \emph{input} (features) variables. Usually we are given a training data set which contains a fixed number $N$ of observations $\{\vec{x_n}\}_{n=1,2,...,N}$ together with their corresponding outcome values $\{y_n\}$. The goal is then to predict the value of $y$ for a new given value of $\vec{x} \in \RR^{m}$. For instance, suppose we have a dataset giving the living areas, number of rooms, the district number location and prices of $50$ departments in Paris, France: \\

\begin{center}
\begin{tabular}{|c|c|c|c|}
	\hline
	Living Area (\si{\square\meter}) &  Number of Rooms &  District &  Price (\euro) \\
	\hline
	57.45  & 3  &  9  &  319000  \\
	47.93  & 2  & 15  &  220000  \\
	59.72  & 3  & 13  &  324000  \\
	72.85  & 4  &  1  &  411000  \\
	46.49  & 2  &  7  &  282000  \\
	46.49  & 2  &  9  &  242000  \\
	73.69  & 4  &  1  &  392000  \\
	61.51  & 3  & 12  &  328000  \\
	42.96  & 2  &  8  &  253000  \\
	58.14  & 3  & 11  &  327000  \\
	43.05  & 2  & 19  &  225000  \\
	43.01  & 2  & 17  &  245000  \\
	53.63  & 3  &  8  &  300000  \\
	21.30  & 1  &  3  &  136000  \\
	24.13  & 1  &  3  &  117000  \\
%	41.57  & 2  &  1  &  221000  \\
%	34.81  & 2  &  5  &  186000  \\
%	54.71  & 3  & 10  &  274000  \\
%	36.38  & 2  &  7  &  208000  \\
%	28.82  & 1  &  9  &  159000  \\
	$\vdots$ & $\vdots$ & $\vdots$ & $\vdots$ \\
\end{tabular}
\end{center}

We can plot this data to observe the Price according to the Living Area:


\begin{center}
	\begin{tikzpicture}
		\begin{axis}[
			width=14cm,  % Increase width
			height=9cm, % Increase height
			title={Paris Apartment Prices: Living Area vs. Price},
			xlabel={Living Area (\si{\square\meter})},
			ylabel={Price (\euro)},
			grid=major,
			colorbar,
			colormap/viridis,
			point meta=min,
			point meta=max,
			scatter/use mapped color={draw=mapped color, fill=mapped color},
			scatter,
			colorbar style={title=Rooms},
			%yticklabel style={/pgf/number format/fixed} % Ensure regular number formatting
			]
			\addplot[
			scatter, only marks,
			mark=x, % Change marker to x
			mark size=2pt,
			scatter src=explicit
			]
			table [meta=Number_of_Rooms] {
				Living_Area Price Number_of_Rooms
				57.45 319000 3
				47.93 220000 2
				59.72 324000 3
				72.85 411000 4
				46.49 282000 2
				46.49 242000 2
				73.69 392000 4
				61.51 328000 3
				42.96 253000 2
				65.32 355000 3
				54.26 290000 3
				50.14 276000 2
				71.67 405000 4
				34.19 180000 1
				45.97 215000 2
				49.03 270000 2
				75.26 420000 4
				39.45 200000 1
				62.58 340000 3
				58.32 310000 3
				41.22 220000 1
				37.54 190000 1
				55.38 295000 3
				66.94 360000 3
				44.81 230000 2
				53.69 285000 3
				67.12 370000 3
				60.25 320000 3
				56.73 300000 3
				38.26 195000 1
				70.85 400000 4
				63.41 345000 3
				43.12 225000 2
				47.78 245000 2
				48.35 250000 2
				64.92 350000 3
				52.13 280000 3
				59.02 315000 3
				42.07 210000 1
				40.88 205000 1
				55.01 290000 3
				68.47 375000 3
				51.96 275000 3
				39.87 198000 1
				44.23 228000 2
				50.78 278000 2
				53.25 283000 3
				61.98 333000 3
				69.73 385000 4
			};
		\end{axis}
	\end{tikzpicture}
\end{center}

%We can also plot the Price against the number of rooms and District: 
%
%\begin{center}
%	\begin{tikzpicture}
%		\begin{axis}[
%			width=14cm,  % Increase width
%			height=9cm, % Increase height
%			title={Paris Apartment Prices: Number of Rooms vs. Price},
%			xlabel={Number of Rooms},
%			ylabel={Price (€)},
%			grid=major,
%			scatter,
%			yticklabel style={/pgf/number format/fixed} % Ensure regular number formatting
%			]
%			\addplot[
%			scatter, only marks,
%			mark=x, % Change marker to x
%			mark size=3pt
%			]
%			table {
%				Number_of_Rooms Price
%				3 319000
%				2 220000
%				3 324000
%				4 411000
%				2 282000
%				2 242000
%				4 392000
%				3 328000
%				2 253000
%				3 355000
%				3 290000
%				2 276000
%				4 405000
%				1 180000
%				2 215000
%				2 270000
%				4 420000
%				1 200000
%				3 340000
%				3 310000
%				1 220000
%				1 190000
%				3 295000
%				3 360000
%				2 230000
%				3 285000
%				3 370000
%				3 320000
%				3 300000
%				1 195000
%				4 400000
%				3 345000
%				2 225000
%				2 245000
%				2 250000
%				3 350000
%				3 280000
%				3 315000
%				1 210000
%				1 205000
%				3 290000
%				3 375000
%				3 275000
%				1 198000
%				2 228000
%				2 278000
%				3 283000
%				3 333000
%				4 385000
%			};
%		\end{axis}
%	\end{tikzpicture}
%\end{center}
%
%\begin{center}
%	\begin{tikzpicture}
%		\begin{axis}[
%			title={Paris Apartment Prices: District vs. Price},
%			xlabel={District Number},
%			ylabel={Price (€)},
%			grid=major,
%			scatter,
%			yticklabel style={/pgf/number format/fixed} % Ensure regular number formatting
%			]
%			\addplot[
%			scatter, only marks,
%			mark=x, % Change marker to x
%			mark size=3pt
%			]
%			table {
%				District Price
%				9 319000
%				15 220000
%				13 324000
%				1 411000
%				7 282000
%				9 242000
%				1 392000
%				12 328000
%				8 253000
%				11 355000
%				19 290000
%				17 276000
%				8 405000
%				3 180000
%				3 215000
%				1 270000
%				5 420000
%				10 200000
%				7 340000
%				9 310000
%				9 220000
%				6 190000
%				4 295000
%				2 360000
%				5 230000
%				8 285000
%				18 370000
%				16 320000
%				14 300000
%				10 195000
%				2 400000
%				20 345000
%				17 225000
%				12 245000
%				11 250000
%				13 350000
%				6 280000
%				15 315000
%				4 210000
%				19 205000
%				5 290000
%				16 375000
%				10 275000
%				14 198000
%				7 228000
%				9 278000
%				20 283000
%				3 333000
%				8 385000
%			};
%		\end{axis}
%	\end{tikzpicture}
%\end{center}

Given data like this, we would like to learn to predict the prices of other apartments in Paris, as a function of the input features $(\text{Living Area},\text{Rooms},\text{District})$. Let us first introduce some notation: We set $\vec{x}^{(i)} = (x_1^{(i)},x_2^{(i)},x_3^{(i)})$ to be the vector in $\RR^3$ whose entries correspond to the features $(\text{Living Area},\text{Rooms},\text{District})$ of the $i$-th apartment in the training set (see Table). For instance, according to the Table 
\[\vec{x}^{(1)}= (57.45, 3, 9)\]

Let $y^{(i)}$ be the output or \emph{target} variable that we are trying to predict; in our case it represents the Price. For instance $y^{(1)} = 319000$. The dataset that we will use to learn consist then of the list of $N=50$ training examples: 
\[\left\{ \left( \vec{x}^{(i)}, y^{(i)} \right) \ : \ i=1,2,...,N \right\}\]
This last list will be then called \emph{training set}. 

\begin{remark}
	In general when constructing a learning problem we have freedom over which features to use so one may decide to collect more data and include other features. We will say more about feature selection later. 
\end{remark}

To achieve the goal of predicting the value of $y$ for new given vectors of $\vec{x} \in \RR^m$, we will formulate a function $y(\vec{x},\vec{w})$ whose values are the predictions for new inputs $\vec{x}$, and where $\vec{w}$ represents a vector of parameters that will be learned from the training set. 

The simplest model for the function $y(\vec{x},\vec{w})$ is the one that consist of a linear combination of the inputs: 

\begin{equation}\label{Eq.Sec.Lin.Reg.1}
	y(\vec{x},\vec{w}) = w_0 + w_1x_1 + \cdots + w_m x_m
\end{equation}

\begin{remark}
	The parameter $w_0$ is usually called a \emph{bias parameter}. The key property of this model is that it is a linear function of the parameters $w_0,...,w_m$ and also of the input variables $x_1,...,x_m$. 
\end{remark}


\subsubsection{Non linear models}

The class of linear models defined by \cref{Eq.Sec.Lin.Reg.1} can be easily extended by considering linear combinations of non-linear functions of the input features $\vec{x}$. More precisely, suppose $\phi_0,...,\phi_m : \RR^m \lto \RR$ are a collection of \emph{basis} functions. Then, we can consider models of the form: 

\[ y(\vec{x},\vec{w}) = w_0 + \sum_{i=1}^{m}w_i\phi_i(\vec{x})\]
For convenience we will set $\phi_0$ to be the constant function $1$ so that the last equation becomes: 

\begin{equation}\label{Eq.Sec.Lin.Reg.2}
	y(\vec{x},\vec{w}) = \sum_{i=0}^{m}w_i\phi_i(\vec{x}) = \vec{w}^T\phi(x)
\end{equation}
where $\vec{w} = (w_0,...,w_m)^T$ and $\phi = (\phi_0,...,\phi_m)^T$.

\begin{remark}
	Models like \cref{Eq.Sec.Lin.Reg.2} are called \emph{linear models} because they are linear in the parameters $\vec{w}$, nevertheless the model $y(\vec{w},\vec{x})$ can be a non-linear function of the inputs $\vec{x}$ (it suffices to consider non-linear basis functions). Moreover, notice that if each $\phi_j$ is defined as the projection onto the $j$-th coordinate then \cref{Eq.Sec.Lin.Reg.2} recovers \cref{Eq.Sec.Lin.Reg.1}. 
\end{remark}

Before deep learning it was common practice in machine learning to use some form of pre-processing of the input variables x, also known as \emph{feature extraction}, expressed in terms of a set of basis functions. The goal was to find a suitable set of basis functions in a way that the resulting learning task could be solved using a simple network model. This is very difficult task. Deep Learning avoids this problem by learning the required nonlinear transformations of
the data from the data set itself. To put it in more simpler terms, Deep Learning can help in the task on finding new suitable features that will help to build a more robust and accurate model. This turns out to be very useful for problems about image recognition. We will see more about this in \cref{part:deep_learning}. 
In the following table we provide a list of some common useful choices of the basis functions (assuming a single feature $x$): 

	% Please add the following required packages to your document preamble:
	% \usepackage{multirow}
	\begin{table}[ht!]
		\centering
		\begin{tabular}{ccc}
			\hline
			\begin{tabular}[c]{@{}c@{}}Basis functions\\ $\phi_j(x)$\end{tabular} &
			\begin{tabular}[c]{@{}c@{}}Model equation\\ $y(x,\vec{w})$\end{tabular} &
			Model nature \\ \hline
			\multirow{2}{*}{$x^j$} &
			\multirow{2}{*}{$w_0 + w_1x + w_2x^2 + \cdots +w_mx^m$} &
			\multirow{2}{*}{Polynomial} \\
			&  &  \\ \hline
			\multirow{2}{*}{$\exp\left( - \frac{(x-\mu_j)^2}{2s^2}\right)$} &
			\multirow{2}{*}{$ \sum w_i \exp\left( - \frac{(x-\mu_j)^2}{2s^2}\right)$} &
			\multirow{2}{*}{Gaussian basis functions} \\
			&  &  \\ \hline
			\multirow{2}{*}{$\sigma\left( \frac{x- \mu_j}{s}\right)$} &
			\multirow{2}{*}{$\sum w_i \sigma\left( \frac{x-\mu_i}{s} \right)$} &
			\multirow{2}{*}{Sigmoidal basis functions} \\
			&  &  \\ \hline
		\end{tabular}
	\end{table}
where $\sigma$ denotes the sigmoid function defined by 
\[ \sigma(z) = \frac{1}{1+ \exp(-z)}\]


\subsubsection{Maximum likelihood}

There are different ways to estimate the parameters $\vec{w}$ for a given model $y(\vec{x},\vec{w})$ using the dataset $\left\{ \left( \vec{x}^{(i)}, y^{(i)} \right) \ : \ i=1,2,...,N \right\}$. Perhaps the most common method consist of finding the parameters $\vec{w}$ that maximize a certain function called \emph{likelihood function}. To derive such function we must make some assumptions about the relation between the target variables $y^{(i)}$ and the inputs $\vec{x}^{(i)}$.

Let us then assume that the target variable $y$ is given by a deterministic function as in \ref{Eq.Sec.Lin.Reg.2} $y(\vec{x},\vec{w})$ with some additive Gaussian noise. This means that the target values and the inputs are related by an equation

\begin{equation}\label{Eq.Sec.Lin.Reg.3} 
y = y(\vec{x},\vec{w}) + \epsilon 
\end{equation}
where $\epsilon$ is a random variable that captures either unmodeled effects (\eg missing important features) or random noise. We assume further that $\epsilon$ is a zero-mean Gaussian with variance $\sigma^2$. We can write this last assumption as $\epsilon \sim \mathcal{N}(0,\sigma^2)$. Therefore, the probability density of $\epsilon$ is given by 

\[ p(\epsilon) = \frac{1}{\sqrt{2\pi}\sigma}\exp\left( - \frac{\epsilon^2}{2\sigma^2}\right)\]
By substituting in \ref{Eq.Sec.Lin.Reg.3} we then get $p(y| \vec{x}; \vec{w}, \sigma) \simeq \mathcal{N}(y(\vec{x},\vec{w}, \sigma))$. In other words,  

\begin{equation}
	p(y| \vec{x}; \vec{w}, \sigma) = \frac{1}{\sqrt{2\pi}\sigma}\exp\left( - \frac{(y-\vec{w}^T\phi(\vec{x}))^2}{2\sigma^2}\right)
\end{equation}
\stnote{There's an error here in this matrix, one has to include phi}
Finally we can then consider the \emph{design matrix} whose rows correspond to the vectors $\vec{x}^{(i)}$ in the dataset together with the vector or target values $\vec{y}$, \ie, 

\[ X = \begin{bmatrix}
	x_{1}^{(1)} & x_{2}^{(1)} & x_{3}^{(1)} & \dots  & x_{m}^{(1)} \\
	x_{1}^{(2)} & x_{2}^{(2)} & x_{3}^{(2)} & \dots  & x_{m}^{(2)} \\
	\vdots & \vdots & \vdots & \ddots & \vdots \\
	x_{1}^{(N)} & x_{2}^{(N)} & x_{3}^{(N)} & \dots  & x_{m}^{(N)}
\end{bmatrix}_{Nxm} \quad \vec{y} = \begin{bmatrix}
y^{(1)} \\
y^{(2)} \\ 
\ddots \\
y^{(N)}
\end{bmatrix}\]


\begin{remark}
	Once we start using our dataset we are working under the assumption that each $y^{(i)}$ satisfies a relation as in \ref{Eq.Sec.Lin.Reg.3} and moreover these of the $\epsilon^{(i)}$ are themselves independent (and hence also the $y^{(i)}$'s given the $\vec{x}^{(i)}$'s).
\end{remark}

From the last assumption we obtain an expression for the likelihood function: 

\begin{equation}
	L(\vec{w}) = L(\vec{w}; X,\vec{y}) = p (\vec{y}| X, \vec{w}, \sigma^2) = \prod_{i=1}^{N} p(y^{(i)}| \vec{x}^{(i)},\vec{w},\sigma)
\end{equation}

The principle of \emph{maximum likelihood} implies that we should choose the parameters $\vec{w}$ so that the data has the highest possible probability. In other words, we should choose $\vec{w}$ so that the maximum likelihood function $L(\vec{w})$ is maximum. To simplify calculations we can, equivalently, maximize the \emph{log likelihood} $\ell(\vec{w}) = \ln (L(\vec{w}))$. A simple calculation shows that

\begin{align}
	\ell(\vec{w}) &= \sum_{i=1}^N \ln \left( \frac{1}{\sqrt{2\pi}\sigma} \exp\left(-\frac{(y^{(i)}- \vec{w}^T\phi(\vec{x}^{(i)}))^2}{2\sigma^2}\right)\right) \\ &= -\frac{N}{2}\ln(2\pi) - \frac{N}{2}\ln(\sigma^2) - \frac{1}{\sigma^2}\cdot \underbrace{\frac{1}{2}\sum_{i=1}^{N} (y^{(i)} - \vec{w}^T\phi(\vec{x}^{(i)}) )^2}_{J(\vec{w})}
\end{align}

The first two quantities in the last equality are constant and therefore, maximizing the log likelihood $\ell(\vec{w})$ is equivalent to minimizing the \emph{sum-of-squares error function}, also called the \emph{cost function} $J(\vec{w})$: 

\begin{equation}\label{Eq.Sec.Lin.Reg.4}
	J(\vec{w}) = \frac{1}{2}\sum_{i=1}^{N} (y^{(i)} - \vec{w}^T\phi(\vec{x}^{(i)}) )^2
\end{equation}

We will observe two ways of minimizing the cost function $J$. 

\subsubsection{Gradient Descent and LMS algorithm}

Recall we want to find the parameters $\vec{w}$ so that the cost function $J(\vec{w})$ is minimized. To do this we can use the \emph{gradient descent} algorithm. The idea is to start with an initial guess, say, $\vec{w} = (w_0,w_1,...,w_m)$ and then iteratively update the parameters using the following rule: 

\begin{equation}\label{Eq.Sec.Lin.Reg.5}
	\vec{w} \leftarrow \vec{w} - \eta \nabla J(\vec{w})
\end{equation} 
where $\eta$ is a (usually small) positive number called the \emph{learning rate} and $\nabla J(\vec{w})$ is the gradient of the cost function $J(\vec{w})$. The gradient is a vector whose components are given by the partial derivatives of $J$ with respect to each parameter $w_j$:
\begin{equation}\label{Eq.Sec.Lin.Reg.6}
	\nabla J(\vec{w}) = \begin{bmatrix}
	\frac{\partial J}{\partial w_0} \\[0.7em]
	\frac{\partial J}{\partial w_1} \\[0.7em]
	\vdots \\
	\frac{\partial J}{\partial w_m}
	\end{bmatrix} = -\begin{bmatrix} 
	\sum_{i=1}^{N} (y^{(i)} - \vec{w}^T\phi(\vec{x}^{(i)}))\phi_0(\vec{x}^{(i)}) \\[0.7em]
	\sum_{i=1}^{N} (y^{(i)} - \vec{w}^T\phi(\vec{x}^{(i)}))\phi_1(\vec{x}^{(i)}) \\[0.7em]
	\vdots \\[0.7em]
	\sum_{i=1}^{N} (y^{(i)} - \vec{w}^T\phi(\vec{x}^{(i)}))\phi_m (\vec{x}^{(i)})
	\end{bmatrix}
\end{equation}

By replacing in \cref{Eq.Sec.Lin.Reg.5} and grouping all coordinates we obtain the following update vector rule for the parameters:

\begin{equation}
	\vec{w} \leftarrow \vec{w} + \eta \cdot \sum_{i=1}^{N} (y^{(i)} - \vec{w}^T\phi(\vec{x}^{(i)}))\phi(\vec{x}^{(i)}) 
\end{equation}

This rule is called the \emph{Least Mean Squares update rule}, sometimes also refered as (LMS) algorithm. The LMS algorithm is a stochastic gradient descent algorithm that updates the parameters $\vec{w}$ using the gradient of the cost function $J(\vec{w})$ at each iteration. The learning rate $\eta$ controls the step size of the update and can be adjusted to improve convergence. The full pseudo code for the LMS algorithm reads as follows: 

\begin{algorithm}[H]
	\caption{Least Mean Squares (LMS) Algorithm}
\begin{algorithmic}[1]
	\Require Training set $\{(\vec{x}^{(i)}, y^{(i)})\}_{i=1}^N$, learning rate $\eta$
	\Require Feature functions $\phi_j$, $j=0,1,\ldots,m$
	\State Initialize $\vec{w} \leftarrow \vec{0}$ or random values
	\Repeat
		\State $\vec{w} \leftarrow \vec{w} + \eta \sum_{i=1}^N (y^{(i)} - \vec{w}^T\phi(\vec{x}^{(i)}))\phi(\vec{x}^{(i)})$
	\Until{convergence or maximum iterations}
	\Return $\vec{w}$
\end{algorithmic}
\end{algorithm}


\begin{remark}
	This method is called \emph{batch gradient descent} because it uses the entire training set to compute the gradient at each iteration. In practice, we can also use \emph{stochastic gradient descent} (SGD) which updates the parameters using only one training example at a time. This is done by replacing the sum in \cref{Eq.Sec.Lin.Reg.5} by a single term $(y^{(i)} - \vec{w}^T\phi(\vec{x}^{(i)}))\phi(\vec{x}^{(i)})$ and iterate over each $i$. The SGD algorithm is usually faster than batch gradient descent and can be used for large datasets. 
\end{remark}

Finally notice also that the LMS algorithm is a special case of the more general \emph{gradient descent} algorithm. The main difference is that in the LMS algorithm we use the gradient of the cost function $J(\vec{w})$ to update the parameters $\vec{w}$, while in gradient descent we can use any function to update the parameters. In general gradient descent is susceptible to capture local minima of a function, while the LMS algorithm will capture one global minimum. This is because the cost function $J(\vec{w})$ is a convex function, which means that it has only one global minimum. In other words the LMS will always converge to the same solution regardless of the initial guess $\vec{w}$. 

\subsubsection{The Normal Equations}
The LMS algorithm is a very powerful method to minimize the cost function $J(\vec{w})$ but it can be slow to converge. In some cases, we can find the optimal solution for the parameters $\vec{w}$ in closed form. This is done by setting the gradient of the cost function $J(\vec{w})$ to zero and solving for $\vec{w}$. This gives us the so-called \emph{normal equations}:


\subsubsection{Regularization}

Let us consider a linear model (of polynomial nature) of the form: 

\begin{equation}\label{Eq.Sec.Lin.Reg.9}
	y(x,\vec{w}) = \sum_{j=0}^{m}w_j\phi_j(x) = w_0 + w_1x + w_2x^2 + \cdots + w_m x^m 
\end{equation}
where $\phi_j(x)= x^j$. Notice that we are considering a single feature $x$ and in this case $m$ is no longer the number of features but the degree of the polynomial $y(x,\vec{w})$. Sometimes this kind of model is also refered as a model having polynomial features $x, x^2 ,x^3,...,x^m$. We will stick to the first terminology for simplicity. 

In this situation we want to use the polynomial model given by \cref{Eq.Sec.Lin.Reg.9} to fit or approximate a dataset $\{ (x^{(i), y^{(i)}}) : i=1,...,N \}$. Notice that we are not using the notation $\vec{x}^{(i)}$ because we are only using one feature $x$. The degree of the polynomial model now becomes a hyperparameter of the model. That is, our goal now is to find the parameters $\vec{w}$ together with the degree $m$ such that the polynomial $y(\vec{w},x)$ is a good approximation of the dataset and such that the model will give suitable predictions for new data. The choice of the degree of the polynomial the becomes important. A high degree polynomial may fit well the dataset but it may not generalize well to new data. This situation is usually know as \emph{overfitting}. Similarly, a low degree polynomial may not fit well the dataset but it may generalize well to new data. This is known as \emph{underfitting}.  

In the following figure we show four examples of the results of fitting polynomials  on some given data points. The polynomials have degrees 1,2,3,4,5 and 7. The data points are shown in blue and the fitted polynomials are shown in red. The first two polynomials (degree 1 and 2) are underfitting the data, while the last two polynomials (degree 5 and 7) are overfitting the data. The polynomial of degree 3 is a good fit for the data. 

\begin{tikzpicture}
    % --- Define the "true" polynomial function ---
    % y = 0.5*x^2 - 2*x + 3
    \pgfmathdeclarefunction{polyfunc}{1}{%
      \pgfmathparse{0.2*(#1)^3 -2*(#1)^2 + (#1) - 4}%
    }

    % --- Parameters ---
    \def\numPoints{11} % Number of points (used conceptually here)
    \def\xmin{0}
    \def\xmax{11}
    \def\noiseLevel{2.5} % Controls the magnitude of the random noise

    \begin{axis}[
        %title={Generated Data for Polynomial Regression},
        xlabel={$x$},
        ylabel={$y$},
        %grid=major,
        %legend pos = north, % Position the legend
        xmin=\xmin-0.5, % Add some padding to axis limits
        xmax=\xmax+0.5,
        % Optional: You might need to adjust ymin/ymax manually
        ymin=-30,
        ymax=30,
    ]

    % --- Plot the generated data points ---
    % We manually define 11 points and add noise to the y-value
    % Noise is calculated as: noiseLevel * (2*rand - 1)
    % (2*rand - 1) gives a uniform random number between -1 and 1
    \addplot[
        only marks, % Only show markers, no connecting lines
        mark=*,    % Use filled circles as markers
        blue,      % Color of the markers
        mark size=1.5pt % Adjust marker size if needed
    ] coordinates {
        % x   { polyfunc(x) + noiseLevel * (2*rand - 1) }
        (0, { polyfunc(0) + \noiseLevel * (2*rand - 1) })
        (1, { polyfunc(1) + \noiseLevel * (2*rand - 1) })
        (2.5, { polyfunc(2.5) + \noiseLevel * (2*rand - 1) })
        (4, { polyfunc(4) + \noiseLevel * (2*rand - 1) })
        (5, { polyfunc(5) + \noiseLevel * (2*rand - 1) })
        ( 7, { polyfunc(7)  + \noiseLevel * (2*rand - 1) })
        ( 7.3, { polyfunc(7.3)  + \noiseLevel * (2*rand - 1) })
        ( 8, { polyfunc(8)  + \noiseLevel * (2*rand - 1) })
        ( 9.1, { polyfunc(9.1)  + \noiseLevel * (2*rand - 1) })
        ( 9.5, { polyfunc(9.5)  + \noiseLevel * (2*rand - 1) })
        ( 10.6, { polyfunc(10.6)  + \noiseLevel * (2*rand - 1) })
    };
    %\addlegendentry{Generated Data Points} % Add entry to legend

    % --- Plot the "true" underlying polynomial function ---
    \addplot[
        domain=\xmin:\xmax, % Range for x
        samples=150,       % Number of points to calculate for a smooth curve
        red,               % Color of the line
        %dashed,            % Style of the line
        thick              % Make the line thicker
    ] {polyfunc(x)}; % Use the function defined earlier
    %\addlegendentry{Best fit $y=3x^3 -2x^2 + x - 4$} % Add entry to legend
    \end{axis}
\end{tikzpicture}



\subsection{Python Implementation}

This section is dedicated to the implementation in python of the LMS algorithm via some problems and exercises taken from Andrew Ng  \textsc{CS229} course at Stanford University. We will use the \texttt{numpy} library which is a powerful library for numerical computing in Python, and the \texttt{pandas} library, a powerful library for data manipulation and analysis. We will also use the \texttt{matplotlib} library to plot the data and the results. In some other problems we will use the \texttt{scikit-learn} library which is a powerful library for machine learning in Python. These notes will contain the main functions and routines in Python. Additional code used to generate the figures and plots will be provided in the \faFolderOpen \texttt{Implementations} folder of the course repository. 

\subsubsection{Cost, Gradient and Gradient Descent}

\subsubsection*{Cost function}
Let us first implement a function that computes the cost function. Remember that the cost function is given by:	

\begin{equation}\label{Eq.Sec.Lin.Reg.8}
	J(\vec{w}) = \frac{1}{2}\sum_{i=1}^{N} (y^{(i)} - \vec{w}^T\phi(\vec{x}^{(i)}) )^2
\end{equation}	
We will assume that the basis functions $\phi_J$ are given by projections onto the $j$-th coordinate, \ie $\phi_j(\vec{x}) = x_j$. So that 

\[ \phi(\vec{x}^{(i)}) = \begin{bmatrix}
    1 & x_1^{(i)} & x_2^{(i)} & \cdots & x_m^{(i)}
\end{bmatrix}^{T} \]
In this case we can write the cost function as in \cref{Eq.Sec.Lin.Reg.1} as follows:

\begin{equation}\label{Eq.Sec.Lin.Reg.9}
	J(\vec{w}) = \frac{1}{2}\sum_{i=1}^{N} (y^{(i)} - \vec{w}^T \vec{x}^{(i)})^2
\end{equation}
where $\vec{x}_0^{(i)}=1$ for all $i=1,...,N$. The code to compute the cost function for a single training example $\vec{x}$ and target value $y$ is then given by: 

\script{Classification}{utils.py}{compute_cost}[caption={Compute cost}, label={code:compute_cost}]

Along all these notes the common conventions and practices will be to use vectorized or Linear algebra notation. In the last code snippet, we start by append to the left a column of ones to the data matrix $X$. This is done to include the bias term $w_0$ in the model. Next we ensure that both the weights vector $\vec{w}$ and the outputs vector $\vec{y}$ are both seen as column vectors. The \texttt{{compute\_cost}} function takes $\vec{w}$ and $\vec{y}$ as 1D numpy arrays which are not the same as $N\times1$ and $(m+1) \times 1$ vectors. Next, we can compute the cost function by simply computing the vector difference \texttt{y - X@w}: 

\[ 
\begin{bmatrix} y^{(1)} \\ y^{(2)} \\ \vdots \\ y^{(N)}\end{bmatrix} - \begin{bmatrix}  
	1 & x_{1}^{(1)} & x_{2}^{(1)} & x_{3}^{(1)} & \dots  & x_{m}^{(1)} \\
	1 & x_{1}^{(2)} & x_{2}^{(2)} & x_{3}^{(2)} & \dots  & x_{m}^{(2)} \\
	\vdots & \vdots & \vdots & \vdots & \ddots & \vdots \\
	1 & x_{1}^{(N)} & x_{2}^{(N)} & x_{3}^{(N)} & \dots  & x_{m}^{(N)}
\end{bmatrix} \cdot \begin{bmatrix} w_0 \\ w_1 \\ \vdots \\ w_m \end{bmatrix} 
\]
Finally, the function simply uses the \texttt{np.sum} function to compute the sum of the squares of the differences. The \texttt{np.sum} function is a vectorized function that computes the sum of all the elements of an array much faster than a for loop.  

\subsubsection*{Gradient}
The next step is to implement the gradient descent algorithm. Recall that we have to update the parameters $\vec{w}$ using the gradient of the cost function $J(\vec{w})$. First, we create a function to compute the gradient:

\script{Classification}{utils.py}{compute_gradient}[caption={Compute Gradient}, label={code:compute_gradient}]

\begin{remark}
	The expression in line 17 of the code snippet is a vectorized version of the expression in \cref{Eq.Sec.Lin.Reg.6}. Formally we have a matrix product of $(\vec{y}- X\cdot \vec{w})^T$ with $X$ . The former expression is a $1xN$ vector and the latter is a $Nx(m+1)$ matrix.  The result is a $1x(m+1)$ vector containing the partial derivatives of the cost function with respect to each parameter $w_j$ or in other words, the gradient of the cost function.
\end{remark}

\subsubsection*{LMS algorithm} 
Finally we proceed to implement the LMS algorithm using gradient descent to minimize the sum-of-squares error function. We will use the \texttt{compute\_cost} and \texttt{compute\_gradient} functions to compute the cost and the gradient at each iteration. 
\script{Classification}{utils.py}{gradient_descent}[caption={Gradient Descent}, label={code:gradient_descent}]

\begin{observation}
It is common practice to consider an average cost function, \ie, we can divide the cost function by $N$ in \cref{Eq.Sec.Lin.Reg.8} and obtain:  

\begin{equation}\label{Eq.Sec.Lin.Reg.10}
	J(\vec{w}) = \frac{1}{2N}\sum_{i=1}^{N} (y^{(i)} - \vec{w}^T \vec{x}^{(i)})^2
\end{equation}
This is done to make the cost function independent of the number of training examples. Of course this has no effect in finding the minimum as $N$ is a constant. However dividing by $N$ makes our calculations more consistent, especially when working with big data sets.  In this case we will have to divide the gradient by $N$ as well. The gradient then turns out to be an "average gradient" which again, makes updates more consistent regardless of the dataset size. We have not done this in the code snippet for simplicity but the reader is encouraged to do so. 
\end{observation}

\subsubsection{House price prediction}

Let us go back to our initial motivating example and use our previous implementation of the LMS algorithm. We will use a dataset of house prices stored in a \texttt{txt} file called \texttt{houses.txt}. The dataset contains prices of 100 houses with $4$ features (size, number of bedrooms, floors and age). This dataset was derived from the \emph{Ames Housing dataset} which is a popular dataset for regression tasks. The dataset is available at \url{https://www.kaggle.com/datasets/shashanknecrothapa/ames-housing-dataset}. 

\begin{remark}
	Each of the problems or practical exercises will have a separate Jupyter notebook where the reader can observe and play around with the code. The reader is encouraged to use the Jupyter notebooks to run the code and play around with the parameters. The Jupyter notebooks also are available in the \faFolderOpen \texttt{Implementations} folder of the course repository. 
\end{remark}

The first four rows of the dataset are shown below: 
\pagebreak
\begin{table}[h] % 'h'ere, 't'op, 'b'ottom, 'p'age - placement options
	\centering % Center the table horizontally
	\caption{House Data Sample} % Add a caption
	\label{tab:house_data_booktabs} % Add a label for cross-referencing
	\begin{tabular}{rrrrrr} % Define columns: r=right-aligned (no vertical lines)
	\toprule % Top rule from booktabs
	% Header Row - using \textbf for emphasis, or just leave plain
	\textbf{Index} & \textbf{Size (sqft)} & \textbf{Number of bedrooms} & \textbf{Number of floors} & \textbf{Age} & \textbf{Price (in 1000's)} \\
	\midrule % Middle rule from booktabs
	% Data Rows
	0 &  952.0 & 2.0 & 1.0 & 65.0 & 271.5 \\
	1 & 1244.0 & 3.0 & 1.0 & 64.0 & 300.0 \\
	2 & 1947.0 & 3.0 & 2.0 & 17.0 & 509.8 \\
	3 & 1725.0 & 3.0 & 2.0 & 42.0 & 394.0 \\
	\bottomrule % Bottom rule from booktabs
	\end{tabular}
\end{table}

We can have a better understanding of our dataset by plotting each of the features against the target variable (price):



	\section{Classification and Logistic Regression}

\subsection{Logistic Regression}

\subsubsection{The perceptron algortihm}

\subsection{Python Implementation}	
	\section{Generalized Linear Models}
%	%%%%%%%%%%%%%%%%%%%%%%%%%%%%%%%%%%%%%%%%
%	%THIRD PART
	\cleardoublepage
	\part{Deep Learning}\label{part:deep_learning}
	\section{Neural Networks}

	\section{Convolutional Neural Networks}\label{sec:cnn}
	\section{Generative Adversarial Networks (GANs)}\label{sec:gans}
	\section{Transformers}\label{sec:transformers}

<<<<<<< HEAD
\subsection{Natural language processing (NLP): Word Embeddings}\label{subsec:nlp_word_embeddings}
=======
\subsection{Natural language processing (NLP): word embeddings}\label{subsec:nlp_word_embeddings}
>>>>>>> bb4ffa3884f86279089af08ae9896dc9ccffb965


\subsection{Attention mechanism}\label{subsec:attention_mechanism}

The fundamental idea behind a transformer model is the \emph{attention} mechanism, which allows the model to focus on different parts of the input sequence when making predictions. This mechanism arose from the need to improve the performance of recurrent neural networks (RNNs) for machine translation tasks \cite{bahdanauNeuralMachineTranslation2016}. Later on, performance was improved considerably by eliminating the RNN architecture altogether and using a fully attention-based architecture, which is the basis of the transformer model \cite{vaswaniAttentionAllYou2017}. 

<<<<<<< HEAD
Let us consider the following three sentences as an example:

\begin{quote}
    \textit{I need to \textbf{run} to catch the bus!}  \\
    \textit{Paul decided to \textbf{run} for president.} \\
    \textit{We had a \textbf{run} of bad luck.} 
\end{quote}

In each case, the word \textit{run} has a different meaning depending on the context. The attention mechanism allows the model to focus on the surrounding words to determine the meaning of \textit{run} in each case. For example, in the first sentence, the model can pay more attention to the words \textit{catch} and \textit{bus}, while in the second sentence, it can focus on \textit{Paul} and \textit{president}.

\subsubsection{Processing}\label{subsubsec:processing}

The input data to a transformer is a collection of vectors $\set{\vec{x}^{(i)}}$ in $\RR^m$ where ${i=1,...,N}$. As it is usual in these notes, each element of the $i$-th vector $\vec{x}^{(i)}_j$ is called a \emph{feature} and the data vectors are called \emph{tokens}. These tokens may correspond to words withing a corpus of text, to a patch of pixels within an image, or to any other type of data sensible to be represented (embedded) as a vector. We will associate a matrix $X \in \RR^{N \times m}$ to the collection of vectors $\set{\vec{x}^{(i)}}$ as follows: 

\begin{equation}
    X = \begin{bmatrix}
       -- (\vec{x}^{(1)})^T -- \\
       -- (\vec{x}^{(2)})^T -- \\
        \vdots \\
       -- (\vec{x}^{(N)})^T --
    \end{bmatrix} 
\end{equation}   

The fundamental block of the Transformer will take the matrix $X$ as input and create a new matrix, $\widehat{X}$ of the same size: 

$$
\widetilde{X} = \textrm{TransformerLayer}(X)
$$

The idea is to create a new matrix $\widetilde{X}$ that contains the same information as $X$, but with the features of each token enhanced by the attention mechanism. We can of course stack (compose) several of these layers in sequence to create a deeper model capable of learning more complex relationships between the tokens. This single transformer layer has two stages: the one acting on columns (features) corresponding to the attention mechanism, and the one acting on rows (tokens) which corresponds to the effect of transforming the features within each token. 


\subsubsection*{Attention}

Let us denote by $\vec{y}^{(1)},...,\vec{y}^{(N)}$ the rows (tokens) of the matrix $\widetilde{X}$. Each of these tokens should live in an embedding space with a richer semantic structure than the tokens of $X$. Since each token in $X$ correspond to some data type (say words) and we want to capture some semantic relation between them, each vector $\vec{y}^{(i)}$ should depend on all the tokens from $X$, \ie, $\vec{x}^{(1)}, \vec{x}^{(2)},...,\vec{x}^{(N)}$. The simplest thing to do is to assume that each $\vec{y}^{(i)}$ depends linearly, or it is linear combination of the tokens in $X$: 

\[ \vec{y}^{(i)} = \sum_{j=1}^N a_{ij} \vec{x}^{(j)} \]

where $\alpha_{ij}$ are the coefficients of the linear combination. These coefficients will be called \emph{attention weights}. We expect these coefficients to be close to zero whenever the input tokens are not relevant to the output token. For instance, in our previous example: \emph{``I need to run to catch the bus''}, the attention weights for the output token associated with the word \textit{catch} should be high for the words \textit{catch}, \textit{to} (second), \textit{bus} and \textit{run} for we need to focus on the object of the action (the bus) and the preceding action (run) to understand the meaning of \textit{catch}. On the other hand, the attention weights should be low for the words \textit{I} and \textit{need} and the first \textit{to}.    

In the following \stnote{add specific ref label to the table} table we present some linguistic intuition about how the weights should be distributed. The first column contains the words of the sentence, the second and third columns contain notation for the input and output tokens, respectively. The fourth column contains the words that should receive low attention weights, while the fifth column contains the words that should receive high attention weights.

\begin{table}[htp] % h=here, t=top, b=bottom, p=page of floats
    \centering{
    %\caption{Intuitive Self-Attention Weights for "I need to run to catch the bus"}
    \label{tab:attention_intuition}
    \begin{tabular}{@{} l c c p{3.5cm} p{3.5cm} @{}} % Use @{} to remove padding at edges
      \toprule % Nicer top line from booktabs
      \textbf{Word} & \textbf{Input Token} & \textbf{Output Token} & \textbf{Low Attention}  & \textbf{High Attention}  \\
      \midrule % Nicer middle line from booktabs
  
      I & $x^1$ & $y^1$ & the, bus, catch & I, need, run \\
      \addlinespace % Add a bit of vertical space
  
      need & $x^2$ & $y^2$ & the, bus & need, I, to (first), run \\
      \addlinespace
  
      to & $x^3$ & $y^3$ & the, bus, I, catch & to (first), need, run \\
      \addlinespace
  
      run & $x^4$ & $y^4$ & I, need & run, to (first), to (second), catch, bus \\
      \addlinespace
  
      to & $x^5$ & $y^5$ & I, need, to (first) & to (second), run, catch, bus \\
      \addlinespace
  
      catch & $x^6$ & $y^6$ & I, need, to (first) & catch, to (second), run, the, bus \\
      \addlinespace
  
      the & $x^7$ & $y^7$ & I, need, to (first), run & the, bus, catch \\
      \addlinespace
  
      bus & $x^8$ & $y^8$ & I, need & bus, the, catch, run \\
  
      \bottomrule % Nicer bottom line from booktabs
    \end{tabular}}
\end{table}

We will then impose the following constraints on the attention weights:

\begin{itemize}
    \item The attention weights are non-negative: $a_{ij} \geq 0$, as we want to avoid situations in which one coefficient can become large and positive while another one compensated by being large and negative. This is not desirable in our case, as we want to focus on the most relevant tokens.
    \item The attention weights sum to one: $\sum_{j=1}^N a_{ij} = 1$. This is a normalization condition that ensures that if an output token pays more attention to one input token, it pays less attention to the others. 
\end{itemize}

\begin{remark}
    Notice that if we assume that the outputs are instead linear combinations of basis functions of the input tokens $\phi_1,...,\phi_N : \RR^m \to \RR$, the two conditions above ensure that the basis functions form a partition of unity. 
    \stnote{Maybe we have to add some more details here. So far it is not that important, just intuition we get from this}
\end{remark}


\subsubsection*{Self-attention}
    Let us discuss how to determine the attention weights $a_{ij}$. The idea first is to use an approach similar to the one used in problems related with information retrieval. The following image shows the basic idea used to find the attention weights: 

    \stnoteil{add image here}


    The main idea is to see each of the input vectors $\vec{x}^{(i)}$ as a \textsc{value} vector that will be used to create the output tokens. We will also use $\vec{x}^{(i)}$ as the \textsc{key} vector for the $i$th input token. Finally  we consider each $\vec{x}^{(j)}$ as \textsc{query} vector for the output $\vec{y}^{(j)}$. To achieve the constraints on the attention weights, we will use a softmax function (with no probabilistic interpretation) so that: 

    \[
    a_{ij} = \frac{\exp((\vec{x}^{(i)})^{ T} \cdot \vec{x}^{(j)})}{\sum_{k=1}^N \exp((\vec{x}^{(i)})^{T} \cdot \vec{x}^{(k)})}
    \]
    By grouping all the output tokens in a single $N \times m$ matrix $Y$ we obtain a nice matrix formula for the output tokens: 
    \begin{equation}\label{Eq.subsec.self_attention.1}
        Y = \textrm{Softmax}(X X^\top) X 
    \end{equation}
    where the softmax function applied on a $N\times N$ matrix $C = [c_{ij}]$ is a new matrix whose entries given by:
    \[ \textrm{Softmax}(C)_{ij} = \frac{\exp(c_{ij})}{\sum_{k=1}^N \exp(c_{ik})} \] 
    This process is called \emph{self-attention} because the same input tokens are used as queries, keys and values. The attention weights are computed by taking the dot product of the query vector with the key vectors, and then applying the softmax function. We will see some variations of this later on.
    
    \subsubsection*{Network parameters}
    So far, the transformation we have described to find the output tokens is fixed in the sense that there are no adjustable parameters and therefore this has no learning capacity from data. We would like to build a network that has some flexibility to choose features to focus on when determining the output tokens. For this, we can start by defining modified feature vectors via a linear transformation to the input tokens through a matrix $U$ of learnable parameters:

    \[ \widehat{X} = X U \]
    where $U$ is an $m \times m$ matrix of learnable weight parameters. Notice that this is analogous to a linear layer in a neural network. By replacing $X$ by $\widehat{X}$ in \eqref{Eq.subsec.self_attention.1} we obtain the following expression for the output tokens:

    \begin{align}\label{Eq.subsec.self_attention.2}
        Y &= \textrm{Softmax}(\widehat{X} \widehat{X}^\top) \widehat{X} \nonumber \\
        &= \textrm{Softmax}(X U U^\top X^\top) X U 
    \end{align}

    \begin{remark}
        This approach has one remarkable characteristic. The matrix $\widehat{X}\widehat{X}^\top$ is symmetric which will in turn imply a symmetric behavior in the attention mechanism. We need much more flexibility, for instance, many tasks in NLP require tokens (words) to be strongly associated with other tokens but not in the opposite sense: The word \textit{hardware} is strongly associated with the word \textit{computer} but the latter may be associated with many other words and its association with the word \textit{hardware} may not be that strong. 
    \end{remark}

    To overcome this limitation we still use the same idea but with independent learnable weight matrices for the query, key and value vectors. We will denote these matrices by $W_q$, $W_k$ and $W_v$ respectively. We then consider: 

    \begin{align*}
        Q &= X W_q  &&\dim W_q = m\times m_k \\ 
        K &= X W_k  &&\dim W_k = m\times m_k \\
        V &= X W_v &&\dim W_v = m\times m_v
    \end{align*}
    where $m_k$ and $m_v$ are the dimensions of the key and value vectors respectively. The dimensions are chosen so that we can perform dot products between the query and key vectors (a typycal choice is to take $m_k= m$). Additionally, $m_v$ will govern the dimension of the output tokens. Finally we obtain an expression for the output tokens as follows:
    \begin{equation}\label{Eq.subsec.self_attention.3}
        Y = \textrm{Softmax}(Q K^\top) V 
    \end{equation}
    
    \subsection{Transformer models}\label{subsec:transformer_models}
=======




\subsection{Transformer models}\label{subsec:transformer_models}
>>>>>>> bb4ffa3884f86279089af08ae9896dc9ccffb965


%	\input{section5}
%	
%	\input{section6}
%	
%	\input{section7}
%	
%	\input{section8}
	%%%%%%%%%%%%%%%%%%%%%%%%%%%%%%%%%%%%%%%%%
	%-------------------------------------------------------------------%
	%-------------------------------------------------------------------%
	%  References                                                       %
	%-------------------------------------------------------------------%
	%-------------------------------------------------------------------%
	
%	\DeclareFieldFormat{labelnumberwidth}{#1}
	\printbibliography
%	\DeclareFieldFormat{labelnumberwidth}{{#1\adddot\midsentence}}
	%\printbibliography[heading=references, notkeyword=alph]
\end{document}	